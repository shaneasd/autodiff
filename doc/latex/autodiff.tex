%           Copyright Matthew Pulver 2018 - 2019.
% Distributed under the Boost Software License, Version 1.0.
%     (See accompanying file LICENSE_1_0.txt or copy at
%           https://www.boost.org/LICENSE_1_0.txt)

\documentclass{article}
\usepackage{amsmath} %\usepackage{mathtools}
\usepackage{amssymb} %\mathbb
\usepackage{color} %\input{variance_triangle.pdf_t}
\usepackage{fancyhdr}
\usepackage[margin=0.75in]{geometry}
\usepackage{graphicx}
\usepackage{hyperref}
%\usepackage{listings}
\usepackage{lmodern} % diagonal slash in \textonehalf instead of horizontal
\usepackage{multirow}
\usepackage{textcomp} % \textonehalf
\usepackage{wrapfig}
\usepackage{xcolor}

\hypersetup{%
  colorlinks=false,% hyperlinks will be black
  linkbordercolor=blue,% hyperlink borders will be red
  urlbordercolor=blue,%
  pdfborderstyle={/S/U/W 1}% border style will be underline of width 1pt
}

\pagestyle{fancyplain}
\fancyhf{} % remove everything
\renewcommand{\headrulewidth}{0pt} % remove lines as well
\cfoot[]{\thepage\\
\scriptsize\color{gray} Copyright \textcopyright\/ Matthew Pulver 2018 - 2019.
Distributed under the Boost Software License, Version 1.0.\\
(See accompanying file LICENSE\_1\_0.txt or copy at
\url{https://www.boost.org/LICENSE\_1\_0.txt})}

\begin{document}

\title{Autodiff\\
\large Automatic Differentiation C++ Library}
\author{Matthew Pulver}
\maketitle

%\date{}

%\begin{abstract}
%\end{abstract}

\section{Synopsis}

\begingroup
\fontsize{10pt}{10pt}\selectfont
\begin{verbatim}
#include <boost/math/differentiation/autodiff.hpp>

namespace boost { namespace math { namespace differentiation {

// Type for variables and constants.
template<typename RealType, size_t Order, size_t... Orders>
using autodiff_fvar = typename detail::nest_fvar<RealType,Order,Orders...>::type;

// Function returning a variable of differentiation.
template<typename RealType, size_t Order, size_t... Orders>
autodiff_fvar<RealType,Order,Orders...> make_fvar(const RealType& ca);

// Type of combined autodiff types.
template<typename RealType, typename... RealTypes>
using promote = typename detail::promote_args_n<RealType,RealTypes...>::type;

namespace detail {

// Single autodiff variable. Independent variables are created by nesting.
template<typename RealType, size_t Order>
class fvar
{
  public:

    // Query return value of function to get the derivatives.
    template<typename... Orders>
    get_type_at<RealType, sizeof...(Orders)-1> derivative(Orders... orders) const;

    // All of the arithmetic and comparison operators are overloaded.
    template<typename RealType2, size_t Order2>
    fvar& operator+=(const fvar<RealType2,Order2>&);

    fvar& operator+=(const root_type&);

    // ...
};

// Standard math functions are overloaded and called via argument-dependent lookup (ADL).
template<typename RealType, size_t Order>
fvar<RealType,Order> floor(const fvar<RealType,Order>&);

template<typename RealType, size_t Order>
fvar<RealType,Order> exp(const fvar<RealType,Order>&);

// ...

} // namespace detail

} } } // namespace boost::math::differentiation
\end{verbatim}
\endgroup

\section{Description}

Autodiff is a header-only C++ library that facilitates the
\href{https://en.wikipedia.org/wiki/Automatic_differentiation}{automatic differentiation} (forward mode) of
mathematical functions of single and multiple variables.

This implementation is based upon the \href{https://en.wikipedia.org/wiki/Taylor_series}{Taylor series} expansion of
an analytic function $f$ at the point $x_0$:

\begin{align*}
f(x_0+\varepsilon) &= f(x_0) + f'(x_0)\varepsilon + \frac{f''(x_0)}{2!}\varepsilon^2 + \frac{f'''(x_0)}{3!}\varepsilon^3 + \cdots \\
  &= \sum_{n=0}^N\frac{f^{(n)}(x_0)}{n!}\varepsilon^n + O\left(\varepsilon^{N+1}\right).
\end{align*}
The essential idea of autodiff is the replacement of numbers with polynomials in the evaluation of $f$. By inputting
the first-order polynomial $x_0+\varepsilon$, the resulting polynomial in $\varepsilon$ contains the function's
derivatives within the coefficients. Each coefficient is equal to a derivative of its respective order, divided
by the factorial of the order.

Assume one is interested in the first $N$ derivatives of $f$ at $x_0$. Then without any loss of precision to the
calculation of the derivatives, all terms $O\left(\varepsilon^{N+1}\right)$ that include powers of $\varepsilon$
greater than $N$ can be discarded, and under these truncation rules, $f$ provides a polynomial-to-polynomial
transformation:

\[
f \qquad : \qquad x_0+\varepsilon \qquad \mapsto \qquad
    \sum_{n=0}^Ny_n\varepsilon^n=\sum_{n=0}^N\frac{f^{(n)}(x_0)}{n!}\varepsilon^n.
\]
C++'s ability to overload operators and functions allows for the creation of a class {\tt fvar} that represents
polynomials in $\varepsilon$. Thus the same algorithm that calculates the numeric value of $y_0=f(x_0)$ is also
used to calculate the polynomial $\sum_{n=0}^Ny_n\varepsilon^n=f(x_0+\varepsilon)$. The derivatives are then found
from the product of the respective factorial and coefficient:

\[ f^{(n)}(x_0)=n!y_n. \]

\section{Examples}

\subsection{Example 1: Single-variable derivatives}

\subsubsection{Calculate derivatives of $f(x)=x^4$ at $x=2$.}

In this example, {\tt autodiff\_fvar<double,5>} is a data type that can hold a polynomial of up to degree 5, and
the {\tt make\_fvar<double,5> x(2.0)} represents the polynomial $2+\varepsilon$.  Internally, this is modeled by
a {\tt std::array<double,6>} whose elements {\tt \{2, 1, 0, 0, 0, 0\}} correspond to the 6 coefficients of the
polynomial upon initialization. Its fourth power is a polynomial with coefficients {\tt y = \{16, 32, 24, 8, 1, 0\}}.
The derivatives are obtained using the formula $f^{(n)}(2)=n!*{\tt y[n]}$.

\begin{verbatim}
#include <boost/math/differentiation/autodiff.hpp>
#include <iostream>

template<typename T>
T fourth_power(T x)
{
    x *= x;
    return x *= x;
}

int main()
{
    using namespace boost::math::differentiation;

    constexpr int Order=5; // The highest order derivative to be calculated.
    const autodiff_fvar<double,Order> x = make_fvar<double,Order>(2); //Find derivatives at x=2.
    const autodiff_fvar<double,Order> y = fourth_power(x);
    for (int i=0 ; i<=Order ; ++i)
        std::cout << "y.derivative("<<i<<") = " << y.derivative(i) << std::endl;
    return 0;
}
/* Output:
y.derivative(0) = 16
y.derivative(1) = 32
y.derivative(2) = 48
y.derivative(3) = 48
y.derivative(4) = 24
y.derivative(5) = 0
*/
\end{verbatim}
The above calculates

\begin{alignat*}{3}
{\tt y.derivative(0)} &=& f(2) =&& \left.x^4\right|_{x=2} &= 16\\
{\tt y.derivative(1)} &=& f'(2) =&& \left.4\cdot x^3\right|_{x=2} &= 32\\
{\tt y.derivative(2)} &=& f''(2) =&& \left.4\cdot 3\cdot x^2\right|_{x=2} &= 48\\
{\tt y.derivative(3)} &=& f'''(2) =&& \left.4\cdot 3\cdot2\cdot x\right|_{x=2} &= 48\\
{\tt y.derivative(4)} &=& f^{(4)}(2) =&& 4\cdot 3\cdot2\cdot1 &= 24\\
{\tt y.derivative(5)} &=& f^{(5)}(2) =&& 0 &
\end{alignat*}

\subsection{Example 2: Multi-variable mixed partial derivatives with multi-precision data type}
\subsubsection{Calculate $\frac{\partial^{12}f}{\partial w^{3}\partial x^{2}\partial y^{4}\partial z^{3}}(11,12,13,14)$
with a precision of about 100 decimal digits,\\
where $f(w,x,y,z)=\exp\left(w\sin\left(\frac{x\log(y)}{z}\right)+\sqrt{\frac{wz}{xy}}\right)+\frac{w^2}{\tan(z)}$.}

In this example, the data type {\tt autodiff\_fvar<cpp\_dec\_float\_100,Nw,Nx,Ny,Nz>} represents
a multivariate polynomial in 4 independent variables, where the highest powers of each are {\tt Nw}, {\tt Nx},
{\tt Ny} and {\tt Nz}. The underlying arithmetic data type, referred to as {\tt root\_type}, is
{\tt boost::multiprecision::cpp\_dec\_float\_100}. The internal data type is
{\tt std::array<std::array<std::array<std::array<cpp\_dec\_float\_100,Nz+1>,Ny+1>,Nx+1>,Nw+1>}. In general, the
{\tt root\_type} is always the first template parameter to {\tt autodiff\_fvar<>} followed by the maximum
derivative order that is to be calculated for each independent variable.

When variables are initialized with {\tt make\_var<...>()}, the position of the last derivative order given in
the template parameter pack determines which variable is taken to be independent. In other words, it determines
which of the 4 different polynomial variables $\varepsilon_w,\varepsilon_x,\varepsilon_y,$ or $\varepsilon_z$
are to be added to the constant term:

\begin{align*}
\texttt{make\_fvar<cpp\_dec\_float\_100,Nw>(11)} &= 11+\varepsilon_w \\
\texttt{make\_fvar<cpp\_dec\_float\_100,0,Nx>(12)} &= 12+\varepsilon_x \\
\texttt{make\_fvar<cpp\_dec\_float\_100,0,0,Ny>(13)} &= 13+\varepsilon_y \\
\texttt{make\_fvar<cpp\_dec\_float\_100,0,0,0,Nz>(14)} &= 14+\varepsilon_z
\end{align*}
Instances of different types are automatically promoted to the smallest multi-variable type that accommodates
both when they are arithmetically combined (added, subtracted, multiplied, divided.)

\begin{verbatim}
#include <boost/math/differentiation/autodiff.hpp>
#include <boost/multiprecision/cpp_dec_float.hpp>
#include <iostream>

template<typename T>
T f(const T& w, const T& x, const T& y, const T& z)
{
  using namespace std;
  return exp(w*sin(x*log(y)/z) + sqrt(w*z/(x*y))) + w*w/tan(z);
}

int main()
{
  using cpp_dec_float_100 = boost::multiprecision::cpp_dec_float_100;
  using namespace boost::math::differentiation;

  constexpr int Nw=3; // Max order of derivative to calculate for w
  constexpr int Nx=2; // Max order of derivative to calculate for x
  constexpr int Ny=4; // Max order of derivative to calculate for y
  constexpr int Nz=3; // Max order of derivative to calculate for z
  using var = autodiff_fvar<cpp_dec_float_100,Nw,Nx,Ny,Nz>;
  const var w = make_fvar<cpp_dec_float_100,Nw>(11);
  const var x = make_fvar<cpp_dec_float_100,0,Nx>(12);
  const var y = make_fvar<cpp_dec_float_100,0,0,Ny>(13);
  const var z = make_fvar<cpp_dec_float_100,0,0,0,Nz>(14);
  const var v = f(w,x,y,z);
  // Calculated from Mathematica symbolic differentiation. See multiprecision.nb for script.
  const cpp_dec_float_100 answer("1976.31960074779771777988187529041872090812118921875499076"
    "582535951111845769110560421820940516423255314");
  std::cout << std::setprecision(std::numeric_limits<cpp_dec_float_100>::digits10)
    << "mathematica   : " << answer << '\n'
    << "autodiff      : " << v.derivative(Nw,Nx,Ny,Nz) << '\n'
    << "relative error: " << std::setprecision(3) << (v.derivative(Nw,Nx,Ny,Nz)/answer-1)
    << std::endl;
  return 0;
}
\end{verbatim}
The relative error between the calculated and actual value is about $6.47\times10^{-99}$.

\section{Mathematics}

In order for the usage of the autodiff library to make sense, a basic understanding of the mathematics will help.

\subsection{Truncated Taylor Series}

Basic calculus courses teach that a real \href{https://en.wikipedia.org/wiki/Analytic_function}{analytic function}
$f : D\rightarrow\mathbb{R}$ is one which can be expressed as a Taylor series at a point
$x_0\in D\subseteq\mathbb{R}$:

\[
f(x) = f(x_0) + f'(x_0)(x-x_0) + \frac{f''(x_0)}{2!}(x-x_0)^2 + \frac{f'''(x_0)}{3!}(x-x_0)^3 + \cdots
\]
One way of thinking about this form is that given the value of an analytic function $f(x_0)$ and its derivatives
$f'(x_0), f''(x_0), f'''(x_0), ...$ evaluated at a point $x_0$, then the value of the function
$f(x)$ can be obtained at any other point $x\in D$ using the above formula.

Let us make the substitution $x=x_0+\varepsilon$ and rewrite the above equation to get:

\[
f(x_0+\varepsilon) = f(x_0) + f'(x_0)\varepsilon + \frac{f''(x_0)}{2!}\varepsilon^2 + \frac{f'''(x_0)}{3!}\varepsilon^3 + \cdots
\]
Now consider $\varepsilon$ as {\it an abstract algebraic entity that never acquires a numeric value}, much like
one does in basic algebra with variables like $x$ or $y$. For example, we can still manipulate entities
like $xy$ and $(1+2x+3x^2)$ without having to assign specific numbers to them.

Using this formula, autodiff goes in the other direction. Given a general formula/algorithm for calculating
$f(x_0+\varepsilon)$, the derivatives are obtained from the coefficients of the powers of $\varepsilon$
in the resulting computation. The general coefficient for $\varepsilon^n$ is

\[\frac{f^{(n)}(x_0)}{n!}.\]
Thus to obtain $f^{(n)}(x_0)$, the coefficient of $\varepsilon^n$ is multiplied by $n!$.

\subsubsection{Example}

Apply the above technique to calculate the derivatives of $f(x)=x^4$ at $x_0=2$.

The first step is to evaluate $f(x_0+\varepsilon)$ and simply go through the calculation/algorithm, treating
$\varepsilon$ as an abstract algebraic entity:

\begin{align*}
f(x_0+\varepsilon) &= f(2+\varepsilon) \\
 &= (2+\varepsilon)^4 \\
 &= \left(4+4\varepsilon+\varepsilon^2\right)^2 \\
 &= 16+32\varepsilon+24\varepsilon^2+8\varepsilon^3+\varepsilon^4.
\end{align*}
Equating the powers of $\varepsilon$ from this result with the above $\varepsilon$-taylor expansion
yields the following equalities:

\[
f(2) = 16, \qquad
f'(2) = 32, \qquad
\frac{f''(2)}{2!} = 24, \qquad
\frac{f'''(2)}{3!} = 8, \qquad
\frac{f^{(4)}(2)}{4!} = 1, \qquad
\frac{f^{(5)}(2)}{5!} = 0.
\]
Multiplying both sides by the respective factorials gives

\[
f(2) = 16, \qquad
f'(2) = 32, \qquad
f''(2) = 48, \qquad
f'''(2) = 48, \qquad
f^{(4)}(2) = 24, \qquad
f^{(5)}(2) = 0.
\]
These values can be directly confirmed by the \href{https://en.wikipedia.org/wiki/Power_rule}{power rule}
applied to $f(x)=x^4$.

\subsection{Arithmetic}

What was essentially done above was to take a formula/algorithm for calculating $f(x_0)$ from a number $x_0$,
and instead apply the same formula/algorithm to a polynomial $x_0+\varepsilon$. Intermediate steps operate on
values of the form

\[
{\bf x} = x_0 + x_1\varepsilon + x_2\varepsilon^2 +\cdots+ x_N\varepsilon^N
\]
and the final return value is of this polynomial form as well. In other words, the normal arithmetic operators
$+,-,\times,\div$ applied to numbers $x$ are instead applied to polynomials $\bf x$. Through the
overloading of C++ operators and functions, floating point data types are replaced with data types that represent
these polynomials. More specifically, C++ types such as `double` are replaced with `std::array<double,N+1>`, which
hold the above $N+1$ coefficients $x_i$, and are wrapped in a `class` that overloads all of the arithmetic
operators.

The logic of these arithmetic operators simply mirror that which is applied to polynomials. We'll look at
each of the 4 arithmetic operators in detail.

\subsubsection arithmetic-addition Addition

Given polynomials $\bf x$ and $\bf y$, how is $\bf z=x+y$ calculated?

To answer this, one simply expands $\bf x$ and $\bf y$ into their polynomial forms and add them together:

\begin{align*}
{\bf z} &= {\bf x} + {\bf y} \\
 &= \left(\sum_{i=0}^Nx_i\varepsilon^i\right) + \left(\sum_{i=0}^Ny_i\varepsilon^i\right) \\
 &= \sum_{i=0}^N(x_i+y_i)\varepsilon^i \\
z_i &= x_i + y_i \qquad \text{for}\; i\in\{0,1,2,...,N\}.
\end{align*}

\subsubsection{Subtraction}

Subtraction follows the same form as addition:

\begin{align*}
{\bf z} &= {\bf x} - {\bf y} \\
 &= \left(\sum_{i=0}^Nx_i\varepsilon^i\right) - \left(\sum_{i=0}^Ny_i\varepsilon^i\right) \\
 &= \sum_{i=0}^N(x_i-y_i)\varepsilon^i \\
z_i &= x_i - y_i \qquad \text{for}\; i\in\{0,1,2,...,N\}.
\end{align*}

\subsubsection{Multiplication}

Multiplication produces higher-order terms:

\begin{align*}
{\bf z} &= {\bf x} \times {\bf y} \\
 &= \left(\sum_{i=0}^Nx_i\varepsilon^i\right) \left(\sum_{i=0}^Ny_i\varepsilon^i\right) \\
 &= x_0y_0 + (x_0y_1+x_1y_0)\varepsilon + (x_0y_2+x_1y_1+x_2y_0)\varepsilon^2 + \cdots +
    \left(\sum_{j=0}^Nx_jy_{N-j}\right)\varepsilon^N + O\left(\varepsilon^{N+1}\right) \\
 &= \sum_{i=0}^N\sum_{j=0}^ix_jy_{i-j}\varepsilon^i + O\left(\varepsilon^{N+1}\right) \\
z_i &= \sum_{j=0}^ix_jy_{i-j} \qquad \text{for}\; i\in\{0,1,2,...,N\}.
\end{align*}
In the case of multiplication, terms involving powers of $\varepsilon$ greater than $N$, collectively denoted
by $O\left(\varepsilon^{N+1}\right)$, are simply discarded. Fortunately, the values of $z_i$ for $i\le N$ do not
depend on any of these discarded terms, so there is no loss of precision in the final answer. The only information
that is lost are the values of higher order derivatives, which we are not interested in anyway. If we were, then
we would have simply chosen a larger value of $N$ to begin with.

\subsubsection{Division}

Division is not directly calculated as are the others. Instead, to find the components of
${\bf z}={\bf x}\div{\bf y}$ we require that ${\bf x}={\bf y}\times{\bf z}$. This yields
a recursive formula for the components $z_i$:

\begin{align*}
x_i &= \sum_{j=0}^iy_jz_{i-j} \\
 &= y_0z_i + \sum_{j=1}^iy_jz_{i-j} \\
z_i &= \frac{1}{y_0}\left(x_i - \sum_{j=1}^iy_jz_{i-j}\right) \qquad \text{for}\; i\in\{0,1,2,...,N\}.
\end{align*}
In the case of division, the values for $z_i$ must be calculated sequentially, since $z_i$
depends on the previously calculated values $z_0, z_1, ..., z_{i-1}$.

\subsection{General Functions}

When calling standard mathematical functions such as {\tt log()}, {\tt cos()}, etc. how should these be written
in order to support autodiff variable types? That is, how should they be written to provide accurate derivatives?

To simplify notation, for a given polynomial ${\bf x} = x_0 + x_1\varepsilon + x_2\varepsilon^2 +\cdots+
x_N\varepsilon^N$ define

\[
{\bf x}_\varepsilon = x_1\varepsilon + x_2\varepsilon^2 +\cdots+ x_N\varepsilon^N = \sum_{i=1}^Nx_i\varepsilon^i
\]
This allows for a concise expression of a general function $f$ of $\bf x$:

\begin{align*}
f({\bf x}) &= f(x_0 + {\bf x}_\varepsilon) \\
 & = f(x_0) + f'(x_0){\bf x}_\varepsilon + \frac{f''(x_0)}{2!}{\bf x}_\varepsilon^2 + \frac{f'''(x_0)}{3!}{\bf x}_\varepsilon^3 + \cdots + \frac{f^{(N)}(x_0)}{N!}{\bf x}_\varepsilon^N + O\left(\varepsilon^{N+1}\right) \\
 & = \sum_{i=0}^N\frac{f^{(i)}(x_0)}{i!}{\bf x}_\varepsilon^i + O\left(\varepsilon^{N+1}\right)
\end{align*}
where $\varepsilon$ has been substituted with ${\bf x}_\varepsilon$ in the $\varepsilon$-taylor series
for $f(x)$. This form gives a recipe for calculating $f({\bf x})$ in general from regular numeric calculations
$f(x_0)$, $f'(x_0)$, $f''(x_0)$, ... and successive powers of the epsilon terms ${\bf x}_\varepsilon$.

For an application in which we are interested in up to $N$ derivatives in $x$ the data structure to hold
this information is an $(N+1)$-element array {\tt v} whose general element is

\[ {\tt v[i]} = \frac{f^{(i)}(x_0)}{i!} \qquad \text{for}\; i\in\{0,1,2,...,N\}. \]

\subsection{Multiple Variables}

In C++, the generalization to mixed partial derivatives with multiple independent variables is conveniently achieved
with recursion. To begin to see the recursive pattern, consider a two-variable function $f(x,y)$. Since $x$
and $y$ are independent, they require their own independent epsilons $\varepsilon_x$ and $\varepsilon_y$,
respectively.

Expand $f(x,y)$ for $x=x_0+\varepsilon_x$:
\begin{align*}
f(x_0+\varepsilon_x,y) &= f(x_0,y)
+ \frac{\partial f}{\partial x}(x_0,y)\varepsilon_x
+ \frac{1}{2!}\frac{\partial^2 f}{\partial x^2}(x_0,y)\varepsilon_x^2
+ \frac{1}{3!}\frac{\partial^3 f}{\partial x^3}(x_0,y)\varepsilon_x^3
+ \cdots
+ \frac{1}{M!}\frac{\partial^M f}{\partial x^M}(x_0,y)\varepsilon_x^M
+ O\left(\varepsilon_x^{M+1}\right) \\
&= \sum_{i=0}^M\frac{1}{i!}\frac{\partial^i f}{\partial x^i}(x_0,y)\varepsilon_x^i + O\left(\varepsilon_x^{M+1}\right).
\end{align*}
Next, expand $f(x_0+\varepsilon_x,y)$ for $y=y_0+\varepsilon_y$:

\begin{align*}
f(x_0+\varepsilon_x,y_0+\varepsilon_y) &= \sum_{j=0}^N\frac{1}{j!}\frac{\partial^j}{\partial y^j}
    \left(\sum_{i=0}^M\frac{1}{i!}\frac{\partial^if}{\partial x^i}\right)(x_0,y_0)\varepsilon_x^i\varepsilon_y^j
    + O\left(\varepsilon_x^{M+1}\right) + O\left(\varepsilon_y^{N+1}\right) \\
&= \sum_{i=0}^M\sum_{j=0}^N\frac{1}{i!j!}\frac{\partial^{i+j}f}{\partial x^i\partial y^j}(x_0,y_0)
   \varepsilon_x^i\varepsilon_y^j + O\left(\varepsilon_x^{M+1}\right) + O\left(\varepsilon_y^{N+1}\right).
\end{align*}

Similarly to the single-variable case, for an application in which we are interested in up to $M$ derivatives in
$x$ and $N$ derivatives in $y$, the data structure to hold this information is an $(M+1)\times(N+1)$
array {\tt v} whose element at $(i,j)$ is

\[
{\tt v[i][j]} = \frac{1}{i!j!}\frac{\partial^{i+j}f}{\partial x^i\partial y^j}(x_0,y_0)
    \qquad \text{for}\; (i,j)\in\{0,1,2,...,M\}\times\{0,1,2,...,N\}.
\]
The generalization to additional independent variables follows the same pattern. This is made more concrete with
C++ code in the next section.

\section{Usage}

\subsection{Single Variable}

To calculate derivatives of a single variable $x$, at a particular value $x_0$, the following must be
specified at compile-time:

\begin{enumerate}
\item The numeric data type {\tt T} of $x_0$. Examples: {\tt double},
    {\tt boost::multiprecision::cpp\_dec\_float\_100}, etc.
\item The maximum derivative order $M$ that is to be calculated with respect to $x$.
\end{enumerate}
Note that both of these requirements are entirely analogous to declaring and using a {\tt std::array<T,N>}. {\tt T}
and {\tt N} must be set as compile-time, but which elements in the array are accessed can be determined at run-time,
just as the choice of what derivatives to query in autodiff can be made during run-time.

To declare and initialize $x$:

\begin{verbatim}
    using namespace boost::math::differentiation;
    autodiff_fvar<T,M> x = make_fvar<T,M>(x0);
\end{verbatim}
where {\tt x0} is a run-time value of type {\tt T}. Assuming {\tt 0 < M}, this represents the polynomial $x_0 +
\varepsilon$. Internally, the member variable of type {\tt std::array<T,M>} is {\tt v = \{ x0, 1, 0, 0, ... \}},
consistent with the above mathematical treatise.

To find the derivatives $f^{(n)}(x_0)$ for $0\le n\le M$ of a function
$f : \mathbb{R}\rightarrow\mathbb{R}$, the function can be represented as a template

\begin{verbatim}
    template<typename T>
    T f(T x);
\end{verbatim}
Using a generic type {\tt T} allows for {\tt x} to be of a regular type such as {\tt double}, but also allows for\\
{\tt boost::math::differentiation::autodiff\_fvar<>} types.

Internal calls to mathematical functions must allow for
\href{https://en.cppreference.com/w/cpp/language/adl}{argument-dependent lookup} (ADL). Many standard library functions
are overloaded in the {\tt boost::math::differentiation} namespace. For example, instead of calling {\tt std::cos(x)}
from within {\tt f}, include the line {\tt using std::cos;} and call {\tt cos(x)} without a namespace prefix.

Calling $f$ and retrieving the calculated value and derivatives:

\begin{verbatim}
    using namespace boost::math::differentiation;
    autodiff_fvar<T,M> x = make_fvar<T,M>(x0);
    autodiff_fvar<T,M> y = f(x);
    for (int n=0 ; n<=M ; ++n)
        std::cout << "y.derivative("<<n<<") == " << y.derivative(n) << std::endl;
\end{verbatim}
{\tt y.derivative(0)} returns the undifferentiated value $f(x_0)$, and {\tt y.derivative(n)} returns $f^{(n)}(x_0)$.
Casting {\tt y} to type {\tt T} also gives the undifferentiated value. In other words, the following 3 values
are equal:

\begin{enumerate}
\item {\tt f(x0)}
\item {\tt y.derivative(0)}
\item {\tt static\_cast<T>(y)}
\end{enumerate}

\subsection{Multiple Variables}

Independent variables are represented in autodiff as independent dimensions within a multi-dimensional array.
This is perhaps best illustrated with examples. The {\tt namespace boost::math::differentiation} is assumed.

The following instantiates a variable of $x=13$ with up to 3 orders of derivatives:

\begin{verbatim}
    autodiff_fvar<double,3> x = make_fvar<double,3>(13);
\end{verbatim}
This instantiates {\bf an independent} value of $y=14$ with up to 4 orders of derivatives:

\begin{verbatim}
    autodiff_fvar<double,0,4> y = make_fvar<double,0,4>(14);
\end{verbatim}
Combining them together {\bf promotes} their data type automatically to the smallest multidimensional array that
accommodates both.

\begin{verbatim}
    // z is promoted to autodiff_fvar<double,3,4>
    auto z = 10*x*x + 50*x*y + 100*y*y;
\end{verbatim}
The object {\tt z} holds a 2-dimensional array, thus {\tt derivative(...)} is a 2-parameter method:

\[
{\tt z.derivative(i,j)} = \frac{\partial^{i+j}f}{\partial x^i\partial y^j}(13,14)
    \qquad \text{for}\; (i,j)\in\{0,1,2,3\}\times\{0,1,2,3,4\}.
\]
A few values of the result can be confirmed through inspection:

\begin{verbatim}
    z.derivative(2,0) == 20
    z.derivative(1,1) == 50
    z.derivative(0,2) == 200
\end{verbatim}
Note how the position of the parameters in {\tt derivative(...)} match how {\tt x} and {\tt y} were declared.
This will be clarified next.

\subsubsection{Two Rules of Variable Initialization}

In general, there are two rules to keep in mind when dealing with multiple variables:

\begin{enumerate}
\item Independent variables correspond to parameter position, in both the initialization {\tt make\_fvar<T,...>}
    and calls to {\tt derivative(...)}.
\item The last template position in {\tt make\_fvar<T,...>} determines which variable a derivative will be
   taken with respect to.
\end{enumerate}
Both rules are illustrated with an example in which there are 3 independent variables $x,y,z$ and 1 dependent
variable $w=f(x,y,z)$, though the following code readily generalizes to any number of independent variables, limited
only by the C++ compiler/memory/platform. The maximum derivative order of each variable is {\tt Nx}, {\tt Ny}, and
{\tt Nz}, respectively. Then the type for {\tt w} is {\tt boost::math::differentiation::autodiff\_fvar<T,Nx,Ny,Nz>}
and all possible mixed partial derivatives are available via

\[
{\tt w.derivative(nx,ny,nz)} =
    \frac{\partial^{n_x+n_y+n_z}f}{\partial x^{n_x}\partial y^{n_y}\partial z^{n_z} }(x_0,y_0,z_0)
\]
for $(n_x,n_y,n_z)\in\{0,1,2,...,N_x\}\times\{0,1,2,...,N_y\}\times\{0,1,2,...,N_z\}$ where $x_0, y_0, z_0$ are
the numerical values at which the function $f$ and its derivatives are evaluated.

In code:
\begin{verbatim}
    using namespace boost::math::differentiation;

    using var = autodiff_fvar<double,Nx,Ny,Nz>; // Nx, Ny, Nz are constexpr size_t.

    var x = make_fvar<double,Nx>(x0);       // x0 is of type double
    var y = make_fvar<double,Nx,Ny>(y0);    // y0 is of type double
    var z = make_fvar<double,Nx,Ny,Nz>(z0); // z0 is of type double

    var w = f(x,y,z);

    for (size_t nx=0 ; nx<=Nx ; ++nx)
        for (size_t ny=0 ; ny<=Ny ; ++ny)
            for (size_t nz=0 ; nz<=Nz ; ++nz)
                std::cout << "w.derivative("<<nx<<','<<ny<<','<<nz<<") == "
                    << w.derivative(nx,ny,nz) << std::endl;
\end{verbatim}
Note how {\tt x}, {\tt y}, and {\tt z} are initialized: the last template parameter determines which variable
$x, y,$ or $z$ a derivative is taken with respect to. In terms of the $\varepsilon$-polynomials
above, this determines whether to add $\varepsilon_x, \varepsilon_y,$ or $\varepsilon_z$ to
$x_0, y_0,$ or $z_0$, respectively.

In contrast, the following initialization of {\tt x} would be INCORRECT:

\begin{verbatim}
    var x = make_fvar<T,Nx,0>(x0); // WRONG
\end{verbatim}
Mathematically, this represents $x_0+\varepsilon_y$, since the last template parameter corresponds to the
$y$ variable, and thus the resulting value will be invalid.

\subsubsection{Type Promotion}

The previous example can be optimized to save some unnecessary computation, by declaring smaller arrays,
and relying on autodiff's automatic type-promotion:

\begin{verbatim}
    using namespace boost::math::differentiation;

    autodiff_fvar<double,Nx> x = make_fvar<double,Nx>(x0);
    autodiff_fvar<double,0,Ny> y = make_fvar<double,0,Ny>(y0);
    autodiff_fvar<double,0,0,Nz> z = make_fvar<double,0,0,Nz>(z0);

    autodiff_fvar<double,Nx,Ny,Nz> w = f(x,y,z);

    for (size_t nx=0 ; nx<=Nx ; ++nx)
        for (size_t ny=0 ; ny<=Ny ; ++ny)
            for (size_t nz=0 ; nz<=Nz ; ++nz)
                std::cout << "w.derivative("<<nx<<','<<ny<<','<<nz<<") == "
                    << w.derivative(nx,ny,nz) << std::endl;
\end{verbatim}
For example, if one of the first steps in the computation of $f$ was {\tt z*z}, then a significantly less number of
multiplications and additions may occur if {\tt z} is declared as {\tt autodiff\_fvar<double,0,0,Nz>} as opposed to \\
{\tt autodiff\_fvar<double,Nx,Ny,Nz>}. There is no loss of precision with the former, since the extra dimensions
represent 0 values. Once {\tt z} is combined with {\tt x} and {\tt y} during the computation, the types will be
promoted as necessary.  This is the recommended way to initialize variables in autodiff.

\section{Function Guidelines}

In order to write a function that accurately calculates derivatives for use with autodiff, there are a couple
of guidelines that will help in getting accurate derivatives:

\begin{itemize}
\item Operate on open intervals of real numbers as a whole, rather than individual points.  Avoid
returning values based on individual point values in the domain.  For example, if an implementation of the
{\tt fourth\_power()} function above has as its first statement {\tt if (x == 2) return 16; } then no
derivatives will be available for $x=2$; instead {\tt y.derivative(i) == 0} for all {\tt i>0}. Each value within a
connected subset of real numbers must be calculated in terms of an analytic function. The extent to which it is not
will be reflected in the inaccuracy of the derivatives.
\item Avoid intermediate singularities in both the calculated value and its derivatives. For example, $y=\sqrt{x}$
followed by $z=y^2$ may appear to be safe for $x\ge0$ however when $x=0$ then all non-zero derivatives of $z$ will
be NaN. This is due to the infinite derivative of $y=\sqrt{x}$ at $x=0$.
\item When constructing a piecewise analytic function, take the average value of each side of the domain for points
on the boundary between analytic functions. For example, the function $g(x) = \max(0,x)$ should be
defined in this manner:
    \begin{verbatim}
        template<typename T>
        T g(const T& x)
        {
            if (x < 0)
                return static_cast<T>(0);
            else if (0 < x)
                return x;
            else
                return 0.5*x;
        }
    \end{verbatim}
This give sensible values for both $g(0)=0$ and $g'(0)=\frac{1}{2}$.
\end{itemize}

\section{Acknowledgments}

\begin{itemize}
\item Kedar Bhat --- C++11 compatibility, codecov integration, and feedback.
\item Nick Thompson --- Initial feedback and help with Boost integration.
\item John Maddock --- Initial feedback and help with Boost integration.
\end{itemize}

\begin{thebibliography}{1}
\bibitem{ad} https://en.wikipedia.org/wiki/Automatic\_differentiation
\end{thebibliography}

\end{document}
